\begin{frame}

    \frametitle{Wer, Wann, Wo?}
  
    \begin{itemize}
      \item Dozent: Prof. Kai-Uwe Sattler
      \item Zielgruppe: Bachelor Informatik
    \end{itemize}

    \begin{columns}
      \begin{column}{0.3\textwidth}
          \begin{center}
           \includegraphics[width=3cm]{fig1/qr-ds1.png}
           \end{center}
      \end{column}
      \begin{column}{0.7\textwidth}
    \begin{itemize}
    \item Übung: Eric Tröbs
      \begin{itemize}
      \item Diskussion und Lösung praktischer Data-Science-Aufgaben
      \end{itemize}
    \item Prüfung: mündlich
    \end{itemize}
  \end{column}
\end{columns}
\end{frame}
  
  %%%%%%%%%%%%%%%%%%%%%%%%%%%%%%%%%%%%%%%%%%%%%%%%%%%%%%%%%%%%%%%%%%%%%%
  
  \begin{frame}
  
    \frametitle{Voraussetzungen \& Ablauf}
  
  \hl{Voraussetzungen}
  \begin{itemize}
  \item Vorlesung: Datenbanksysteme (Bachelor)
  \item Grundkenntnisse: Statistik, SQL, Python(?) \dots
  \end{itemize}


  \hl{Ablauf}
  \begin{itemize}
    \item "`hybride"' Vorlesung: Präsenz + Flipped Classroom für ausgewählte Teile
    \item Übungen mit praktischen (Haus-)Aufgaben (Jupyter, Einreichung über Moodle)
    \item Miniprojekt
    \end{itemize}
  
  \end{frame}
  
  %%%%%%%%%%%%%%%%%%%%%%%%%%%%%%%%%%%%%%%%%%%%%%%%%%%%%%%%%%%%%%%%%%%%%%
  
  \begin{frame}
  
    \frametitle{Lernziele}
  
  \begin{itemize}
  \item Überblick über \hl{Grundlagen und Verfahren} von Data Science
  \item Fähigkeit 
  \begin{itemize}
  \item zum grundlegenden Umgang mit Daten ("`Data Literacy"')
  \item zur Auswahl, Bewertung und Einsatz konkreter Verfahren für
    gegebene Anwendung 
  \end{itemize}
  \item Kompetenzen in der \hl{praktischen Arbeit} mit Data-Science-Themen 
\end{itemize}
  
  \end{frame}

%%%%%%%%%%%%%%%%%%%%%%%%%%%%%%%%%%%%%%%%%%%%%%%%%%%%%%%%%%%%%%%%%%%%%%
  
  \begin{frame}
    \frametitle{Überblick}
  
    \begin{enumerate}
    \item<1-> Einführung: Was ist Data Science?
    \item<2-> Datenanalyse: Daten, Prozess, Aufgaben
    \item<3-> Werkzeuge zur Datananalyse: SQL, Python, Jupyter
    \item<4-> Datenbeschaffung, -vorbereitung und -bereinigung
    \item<5-> Datenvisualisierung
    \item<6-> Data Warehousing und OLAP
    \item<7-> Ausgewählte Analyseverfahren: Regression, Klassifikation, Clustering
    \item<8-> Analyse von Graphen
    \item<9-> Textanalyse
   \end{enumerate}
  
  \end{frame}

%%%%%%%%%%%%%%%%%%%%%%%%%%%%%%%%%%%%%%%%%%%%%%%%%%%%%%%%%%%%%%%%%%%%%%
  
\begin{frame}

\frametitle{Mini-Projekt}

\begin{itemize}
\item Ziel: Erprobung der Kenntnisse aus Vorlesung und Übung
\item 4 Projekte zur Auswahl
\item 1 bis 2 Pflichtaufgaben
\item selbstständiges Finden einer Wahlaufgabe
\item Fokus auf Erkundung von unbekannten Daten, Anwenden von gelernten Methoden und Darstellung der Ergebnisse
\item \textbf{\color{red} auf freiwilliger Basis}
\item \textbf{\color{red} eigener Datensatz nach Abstimmung möglich}
\end{itemize}

\end{frame}

%%%%%%%%%%%%%%%%%%%%%%%%%%%%%%%%%%%%%%%%%%%%%%%%%%%%%%%%%%%%%%%%%%%%%%
  
\begin{frame}

\frametitle{Mini-Projekt}

\begin{enumerate}
\item Mensen {\tiny HTML, Mustersuche}
\item TMDb {\tiny JSON API, Mustersuche}
\item Treibstoffpreise {\tiny CSV, Mustersuche}
\item Twitter {\tiny Text mit Metadaten, Textanalyse}
\end{enumerate}

\end{frame}
  
  %%%%%%%%%%%%%%%%%%%%%%%%%%%%%%%%%%%%%%%%%%%%%%%%%%%%%%%%%%%%%%%%%%%%%%
  
  
  \begin{frame}
    \frametitle{Weitere Literatur}
  
    \begin{thebibliography}{Ester, 2000}
     \bibitem[VanderPlas, 2018]{VanderPlan, 2018}
     J. VanderPlas.
     \newblock {\em Data Science mit Python}. 
     \newblock mitp-Verlag, 2018.  
     \bibitem[Herbold, 2022]{Herbold, 2022}
     S. Herbold.
     \newblock {\em Data Science Crashkurs}. 
     \newblock dpunkt-Verlag, 2022.    
     \bibitem[Saake, 2018]{Saake, 2018}
     G. Saake, K. Sattler, A. Heuer.
     \newblock{\em Datenbanken --- Konzepte und Sprachen}
     \newblock 6. Auflage, mitp-Verlag, 2018.
     \bibitem[Köppen, 2014]{Köppen, 2014} 
     V. Köppen, G. Saake, K. Sattler.
     \newblock {\em Data Warehouse Technologien}. 
     \newblock 2. Auflage, mitp-Verlag, 2014.    

  %\bibitem[Zhao, 2013]{Zhao, 2013}
  %  Y. Zhao.
  %  \newblock {\em R and Data Mining: Examples and Case Studies}. 
  %  \newblock Elsevier, 2013.  
  %  \newblock \url{http://www.rdatamining.com}.
  \end{thebibliography}
  
  \end{frame}
  
